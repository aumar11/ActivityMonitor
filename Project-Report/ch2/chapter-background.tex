\chapter{Background} % (fold)
\label{cha:chapter_background}


\section{Smartphones} % (fold)
\label{sec:section_smartphones}
The modern smartphone allows not only for mobile computing and communication but has been equipped with many built in useful sensors. However there have been four main factors which have led to the development of the area of research called mobile phone sensing~\cite{Campbell2010sensing} . The first factor is how readily available embedded sensors are to be incorporated with a smartphone. The second involves the programmable nature of the modern smartphone, the tools available for third party applications to be programmed and installed has intensified. The development of the app store has allowed for these third party programs to be delivered to all corners of the world, increasing development popularity and almost creating a new job titled (mobile app developer). Lastly the concept of mobile cloud computing has enabled the application computation to be done on a central server. This means that the capability and scalability of an application developed for a smartphone has increased. 


\section{Smartphone As Proxy For User?} % (fold)
\label{sec:section_proxy}
The ubiquitous nature of a modern smartphone has brought rise to the notion that it can act as reliable proxy for its user.  This interesting concept, if assumed true suggests that contextual information about the mobile can deduce contextual information about its user. Thus expanding the applications that can be developed for a mobile phone vastly. Roy Want argues in his report ``You Are Your Cell Phone''~\cite{want2008cellphone}
\begin{quote}
Now that cell phones have become mobile and ubiquitous computers in their own right, we can take the proxy concept to a new  level. [...] The personal-proxy analogy becomes even stronger when you consider adding sensors to a phone to support mobile  context-aware operation. A person’s cell phone experiences almost all the physical parameters that the person experiences—it feels the same forces, travels at the same velocity, is about the same temperature, is exposed to the same sounds and pollution levels, and is near the same people and equipment.
\end{quote}
Patel et al. however makes the point that a mobile phone may not be as accurate a proxy for its user as first assumed.  In``Farther Thank You May Think. An Empirical Investigation of the proximity of Users to Their Mobile Phones''~\cite{patel2006farther} he states, 
\begin{quote}
Many researchers and application designers make the implicit assumption that people are likely to have their mobile phones with them and available most of the time. However, little empirical evidence on the actual proximity relationship between a mobile phone and its owner exists.
\end{quote}
He then undertakes a study of sixteen users to quantify the physical relationship between the mobile phone and its user. In this investigation users were made to wear a Bluetooth beacon around their necks most of the time. This beacon would act as a more guaranteed proxy for the location of its user. It was devised to gather data about the distance between the beacon and the user mobile phone. This was calculated using the signal strength of the response of the phone. The study’s findings,
\begin{quote}
demonstrated that people often have higher expectations of their own proximity and availability to their mobile phones than is accurate in reality.
\end{quote}
The mobile phone was found to be within arm’s reach an average of 58\% of the
time. The experiment also found that the proximity of the phone to the user varied when away from home (70\%) as opposed to when at home (30\%).


\section{Accelerometer} % (fold)
\label{sec:section_accelerometer}
Acceleration is defined as the rate of change of velocity with respect to time a = dv/dt. It is measured in meters per second per second. A `g' is a measurement of acceleration this is equivalent to earth’s gravity at sea level~\cite{tiacc}.
\paragraph{}
An accelerometer is an electromagnetic device that is capable of measuring acceleration forces. The forces that an accelerometer can measure are categorised into two groups, Static forces and dynamic forces. Static forces include the constant force of gravity and dynamic forces that are caused when the accelerometer is moved. The accelerometer is not just limited to measuring acceleration, with this information it can measure Tilt and tilt angle, rotation, vibration, collision and gravity~\cite{lindsay2005parallax}.
\paragraph{} % (fold)
The accelerometer allows for both the orientations and the direction to be calculated, and thus the orientation and direction of any device that it is connected to i.e. a mobile phone.  This allows for assumptions to be produced of the devices surroundings.


\section{Android Coordinate System}
\label{sec:coordinate}
The values the accelerometer, amongst other sensors, outputs are defined with accordance to the standard coordinate system used by android~\cite{android2013coordinate}. It is defined relative `to the device's screen when the device is held in its default orientation'~\cite{android2013coordinate}, see figure~\ref{fig:Figures_coordinate}.
\begin{figure}[h]
  \includegraphics[width=\textwidth]{figures/coordinatefig.pdf}
  \caption{Android Coordinate System.}
  \label{fig:Figures_coordinate}
\end{figure} 
\begin{quote}
When a device is held in its default orientation, the X axis is horizontal and points to the right, the Y axis is vertical and points up, and the Z axis points toward the outside of the screen face. In this system, coordinates behind the screen have negative Z values.
\end{quote}
It is essential to remember that the coordinate system does not change when the orientation of the screen changes.

\section{Analysis of Accelerometer Data Background}
\label{sec:accanalback}

There has been a large interest in activity recognition and is becoming a popular research area. In particular data gathered by accelerometers on users has shown promising results~\cite{bao2004annotated}~\cite{ravi2005recognition}~\cite{randell2000awerness}. In terms of machine learning, activity recognition is formulated as a classification problem~\cite{ravi2005recognition}. Many researchers have used ``windows, window sizes and overlapping windows of the data''\cite{ravi2005recognition} to categorise activities. However, in their article Hristijan Gjoreski et al. demonstrate that this approach is not a viable option when determining whether the user is a sedentary state, 
\begin{quote}
With this approach, every instance cannot be classified, but a group of instances (one window) are classified together. The disadvantage of this approach is that short activities like going down and standing up cannot be recognized.
\end{quote}
They go on to state that for the scenario that every activity is to be classified that the key features used in the classification are
\begin{quote}
Some of the attributes are the traditional features that are used for acceleration activity recognition like: mean, standard deviation, root mean square, etc., but there are also some features that measure the direction and the movement of the person.
\end{quote}