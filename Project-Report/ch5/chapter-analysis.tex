\chapter{Analysis} % (fold)
\label{cha:chapter_analysis}



\section{Why Jeans Pocket?} % (fold)
\label{sec:section_pocket}
Thigh accelerometer reference look at conclusion......
\paragraph{}
For this early stage in the apps development it was essential to assume that the smartphone wis placed in the users front jeans pocket. This position is assumed for two reasons. Firstly it is natural for the user to place the smartphone in their front pocket() whether it be left or right[need to find backing]. Secondly in his article look above paragraph they placed several accelerometer all around participants, one of which was placed on the users thigh. Placing ones phone in their pocket places the accelerometer in a similar if not identical position in terms of how the accelerometer is expected to respond. 
\paragraph{}
An experiment was conducted to graphically represent the responses of the accelerometer whilst the smartphone was placed in the users left pocket, right pocket and top pocket. The user placed the smartphone in the respective pockets sat down on a seat and began recording. After five seconds had past the user stood up, once fully standing they waited a further 5 seconds before sitting back down on the chair and remained in a sedentary state for a further 5 seconds before stopping the recording. The results of this experiment revealed two key conclusions. Firstly if the user places the smartphone in either the left or right pocket the gathered data would respond in a similar manner. Secondly when placed in these pockets the response can be categorized as a step up and step down signal making it possible to determine the starting state of the user with no prior knowledge. This is not possible for the results shown by the top pocket. When visually represented the signal can be easily broken in to the activities used to create it. 

\subsection{How Long Does Sit/Stand transition last?} % (fold)
\label{sub:subsection_stslength}
To detect a sit to stand several key questions were posed. Firstly how long does a sit to stand last, both in term of time and number of samples. Two test were carried out to obtain this information. Six test subjects were used. For the first test he participant began in a an sedentary state. As soon as the headset button was pressed to start gathering data the user stood up and as soon as they were in this active state they pressed the headset button for a second time to stop recording data. This test was repeated five times by each participant. The second test followed a similar pattern however the participants began in a n active (standing) state and ended in the sedentary state. The averages of these results are shown below. 


\section{Another Section} % (fold)
\label{sec:another_section}


