\chapter{Development} % (fold)
\label{cha:chapter_development}

The design of this system can be split into three significant stages. These are the Android application gathering data,the central server for storing the data and Matlab for processing the data. 

\section{Android Application} % (fold)
\label{sec:section_app}

\subsection{Overview of the functionality} % (fold)
\label{sub:subsection_appoverview}
As mentioned in the introduction, the main aim of the app is to detect when the user is in an active of sedentary state solely by the data gathered by the embedded accelerometer. Currently the app developed is a prototype which allows user to effectively record and store training accelerometer data which will be used in its ultimate goal.

\subsection{Android Fundamentals} % (fold)
\label{sub:android_fundamentals}
\begin{itemize}
  \item Activity: `An activity represents a single screen with a user interface'
  \item Service: `A service is a component that runs in the background to perform long-running operations or to perform work for remote processes.'
  \item BroadCastReceiver: `A broadcast receiver is a component that responds to system-wide broadcast announcements'.
\end{itemize}
\subsection{Main Activity} % (fold)
\label{sub:subsection_mainactivity}
The ``ActivityGatherer'' class is the activity which is launched when the application is first run. A activity simply ``represents a single screen with a user interface''~\cite{android2013fundamentals}. The current version of the app requires a simple GUI as its main purpose is to gather training data. Figure~\ref{fig:main-activity} shows the the starting activity.
\begin{figure}[h]
  \includegraphics[width=\textwidth]{figures/screenshots/main.pdf}
  \caption{Main Activity.}
  \label{fig:main-activity}
\end{figure}
As can be seen in the above figure the main activity contains a spinner widget. The spinner widget is androids equivalent of a drop down menu. The menu choices given by the spinner allows the user to select a label for what type of activity is going to be recorded. 
\begin{figure}[h]
  \includegraphics[width=\textwidth]{figures/screenshots/types.png}
  \caption{Activity Label Type Options.}
  \label{fig:type-choice}
\end{figure}
To keep track of the current spinner item selected the SpinnerActivityTypeListener class was created. This class implements the OnItemSelectedListener and over rides the OnItemSelected() method. In this method the current Preferences of the app are altered according to the label name selected. Preferences is a helper class which is used to store the current state of the app can be accessed globally with in the context of the application.
\paragraph{}
Initial versions of the gui contained a button widget which was used to start and stop the collection of accelerometer data. However, as this data was being used to represent purely the state of the user  assuming the phone was in their pocket an on screen button implies the phone beings out with the users pocket. This then adds noise to both the start and the end of the signal, which at this stage of the application was vital to remove. 
\paragraph{}
This first solution adopted was to introduce a count down timer when the button is first pressed to recored data. This gave the user enough time to place the smartphone into their pocket before the data started recording. However, this solution only eliminated the noise from the beginning of the signal as the user still had to remove the phone out of their pocket to stop recording. The second solution again used count down timers. THe first count down timer was used in a similar fashion to the first solution. The second timer was used to automatically stop the recording after a specified amount of time. This solved the problem of noise before and after the required signal. The disadvantage of this method was that it was not clear how long of a window was required for each type of activity. The final solution took advantage of the button found on the standard Galaxy Nexus headset. 

\subsection{HeadSetReceiver} % (fold)
\label{sub:subsection_headset}
When the main activity is created a HeadSetReceiver is registered. This class implements a BroadcastReciever which listens for a specific type of intent, ACTION MEDIA BUTTON. When the media button is clicked the onReceive() method is invoked. THis method is crucial to creating both the services used to gather accelerometer data and to sync the data with the central server. The headsetreceiver is unregistered once the application is stopped.

\subsection{Main Service} % (fold)
\label{sub:main_service}
The ActivityGathererService when created registers the key listeners/receiver. These are the AccelerometerListener and SyncAlarmReceiver. When the app is not on screen the activity goes onPause() however the service remains to run in the background, this is vital to ensure the app is still gathering data whilst the user can still the smartphone.

\subsection{Gathering Data} % (fold)
\label{sub:data_gathering}
The first stage of the app is to collect the embedded accelerometer output from the phone. The method to do this on Android is by implement SensorEventListener class as well as registering it ot a SensorManager. 
\paragraph{}
The SensorManager is used to access the sensors on the smartphone. The registerListener method is used to register a SensorEventListener to listen for events from a specific sensors at a specified rate. The accelerometer is the sensor required for this project. A rate of 10Hz was used.The documentation of the sensormanger~\cite{android2013manager} states
\begin{quote}
The rate sensor events are delivered at. This is only a hint to the system. Events may be received faster or slower than the specified rate. Usually events are received faster.
\end{quote}
This will be discussed in more detail in later chapters.
\paragraph{}
The AccelerometerListener was created to listen for sensor events. Implemented this class requires the onAccuracyChanged() and more importantly the onSensorChanged() to be over ridden. As discussed previously SensorManager specifies a rate at which sensor events should occur, onSensorChanged() is the method called when the sensor events are received. THe information we are interested in when listing for accelerometer output is Event.value. It is a float array that contains x, y and z axis accelerometer data in value[0], value[1] and value[2] respectively. 

\subsection{DataBase} % (fold)
\label{sub:data_gathering}
This app is used to gather training data to be analyzed. It is therefore essential that this data is stored securely and in a manner that it could be searched through efficiently. The appropriate android solution to this is to create and maintain an SQLite3 database. There are three main classes used in this to create and maintain this database. These are the SampleDBHelper, SampleDb and Sample. 
\subsubsection{SampleDBHelper} % (fold)
\label{ssub:db_helper}
SampleDBHelper implements SQLiteOpenHelper and is specifically used to create and upgrade the database. It is also used to store key database information such as the name of the database, version, table names and table column names. 
\paragraph{}
In the app there are two tables, sample and activity. NEED TO INPUT DATABASE SCHEME.
\paragraph{}
Activity is used to create a unique id and name for each new activity recording. This id is used to query sample to get all accelerometer samples for that unique recording. Type in sample is used to collectively query for all types of activities etc.
\subsubsection{SampleDB} % (fold)
\label{ssub:db}
SampleDB is used to add to and query the database with the help of SampleDBHelper. The three main methods in this class are addSample(), getLatestSampleID() and getlatestActivityID(). AddSAmple is used on the smartphone side to add a new accelerometer sample to the SQlite3 database. The other two methods are used in conjunction with the SyncService class.
\subsubsection{Sample} % (fold)
\label{ssub:sample}
The Sample class was created as a helper class. An instance of this class is used to store a given accelerometer reading with some additional information The accelerometer data is received from Event.value. So a samples stores x, y and z axis accelerometer data. It also stores a timestamp of when the reading was created as well as the label type i.e sit to stand, stand to sit etc. A new sample is created and stored in the SQLite3 database when onSensorChangeEvent is called. db.addnewSample(...).

\subsection{Syncing with Central Server} % (fold)
\label{sub:synching}
It is vital for the app to send the data that has been collected and stored in the SqLite3 database. The classes dedicated to this task are the SyncAlarmReceiver and SyncService.
\subsubsection{SyncAlarmReceiver} % (fold)
\label{ssub:alarm}
SyncAlarmReceiver implements BroadcastReceiver. An Alarm can be set to go off at a specific time trigger, which is received by the on Receive method off broadcastReciver. THis method ensure that the server is continuously updated at regular time intervals.
\paragraph{}
SynchService is the most vital class in this process. When the SyncService is created a new Thread is created. This ensures that data is still gathered by the AccelelerometerListener and put into the on phone database whilst the sync is in progress concurrently parallelism. In the new thread run()is invoked and these are the several key steps. 
\paragraph{}
FIrst establish connection with the server and request for the id of the latest record on the activity and sample table on the MySQL database. Get the all the records that still have to be synced and create a CSV file and GZIP up, and update the server with this new information. The id's of the latest ensure that only the relevant data to be sent is sent. zipping the file compresses the data to be sent, therefore reducing the time taken to send file to the server and free up phone resources. 

\section{Central Server} % (fold)
\label{sec:section_server}
Xampp is an easy to install apache distribution that contains apache, MYSQL, PHP, and Perl amongst other tools. The apache webserver is used to communicate over HTTP with he android. RESTFUL interface thing.
MySQl database was created with the same database schema names included as the SQlite3 database on the android phone using phpmyadmin. Helper PHP scripts were used to query the SQL database to find the latest id of the records stored to work with SyncService described previously via get post. Another php script was used to collect the zipped file sent by the phone and unzip and extract its contents. Once the data is extracted a query is used to insert the contents of the file appropriately in to the MySQL database.

\section{Matlab} % (fold)
\label{sec:section_matlab}
Matlab was used to analyze the data that has been gathered. It was chosen over other data analyze tools mainly due to the ease at which it connected with the Xampp server and MySQL database but without compromising with the level of analysis that could be achieved. One of the main advantage of Matlab is that scripts can be produced. This means once code has been appropriately created and tested it can stored in a script. Then instead of having to write all the code, the name of the script can be entered and all the work is automated. 
 